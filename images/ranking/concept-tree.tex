\begin{tikzpicture}[->,>=stealth',every node/.style={draw,circle},node distance=2cm]
\node (0) {$v_0$};
\node[below left of = 0] (1) {$v_1$};
\node[below of = 0] (2) {$v_2$};
\node[below right of = 0] (3) {$v_3$};
\node[below of = 3] (4) {$v_4$};
\node[below of = 2] (5) {$v_5$};

\path[every node/.style={font=\footnotesize,fill=white}]
(0) edge node {a} (1)
 edge node {b} (2)
 edge node {c} (3)
(3) edge node {d} (4)
(2) edge node {d} (5)
 edge node {d} (4)
;
\end{tikzpicture}