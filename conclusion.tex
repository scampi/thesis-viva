\section{Conclusion}

\frame{
    \frametitle{Summary And Directions for Future Research}
    \begin{columns}[c]
        \column{.3\textwidth}
        \begin{figure}
            \resizebox{.9\textwidth}{!}{
                \begin{tikzpicture}
                    \begin{axis}[
                            ybar,
                            symbolic x coords={Nodes,Edges},
                            enlargelimits=0.25,
                            x=2cm,
                            bar width=0.3cm,
                            xtick=data,
                            ymajorgrids = true,
                            legend style={at={(0.5,-0.15)},
                            anchor=north,legend columns=1},
                        ]
                        \addplot coordinates {
                                (Nodes, 65042837)
                                (Edges, 233051608)
                            };

                        \addplot coordinates {
                                (Nodes, 1450687)
                                (Edges, 19725084)
                            };
                        \legend{DBpedia 2013 (en),Types Summary}
                    \end{axis}
                    %\draw[step=1cm,gray,ultra thick] (0,0) grid (3,5);
                    \draw[red,ultra thick] (.4,2.2) edge[->,>=stealth'] node[above right] {\textbf{$96\%$}} (.7,1.1);
                    \draw[red,ultra thick] (2.4,4.8) edge[->,>=stealth'] node[above right] {\textbf{$92\%$}} (2.7,1.4);
                \end{tikzpicture}
            }
        \end{figure}
        \column{.7\textwidth}
        \begin{description}
            \item<2->[Summary Updates:] Efficiently updates the summary of a dataset without re-doing the process from scratch.
            \item<3->[Data Quality:] Improve the data modeling by identifying structural inconsistencies.
            \item<4->[Approximate Summaries:] Develop summarisation relations which have little impact on the connectivity precision, and yet generate small summaries.
        \end{description}
    \end{columns}
}

\frame[shrink]{
    \frametitle{Contributions}
    {
        \vspace{1cm}
        \scriptsize
        \begin{block}{Graph Summarization:}
            \begin{itemize}
                \item \textbf{Introducing RDF Graph Summary with application to Assisted SPARQL Formulation}, \textit{WebS Workshop, DEXA2012}
                \item \textbf{Efficiency and precision trade-offs in graph summary algorithms}, \textit{Proceedings of the 17th International Database Engineering Applications Symposium}
                \item \textbf{Live SPARQL auto-completion}, \textit{Proceedings of the 2014 International Conference on Posters Demonstrations Track}
            \end{itemize}
        \end{block}
        \begin{block}{Ranking:}
            \begin{itemize}
                \item \textbf{The Sindice-2011 dataset for entity-oriented search in the web of data}, \textit{1st international workshop on entity-oriented search (EOS)}
                \item \textbf{Sindice BM25F at SemSearch 2011}, \textit{Proceedings of the 4th International Semantic Search Workshop}
                \item \textbf{Effective retrieval model for entity with multi-valued attributes: Bm25mf and beyond}, \textit{International Conference on Knowledge Engineering and Knowledge Management}
            \end{itemize}
        \end{block}
    }
}
% vim: et:sw=4
