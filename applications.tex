\section{Applications}

\subsection{Assisted SPARQL Query Editor}

\frame{
    \frametitle{Assisted SPARQL Query Editor}
    \begin{block}{Application}
        \begin{itemize}
            \item Provide suggestions for a missing element in a SPARQL query
            \item Possible suggestions:
                \begin{itemize}
                    \item Graph
                    \item Type
                    \item Attribute
                    \item Relation between two triple patterns
                \end{itemize}
        \end{itemize}
    \end{block}
    \begin{block}{Solution}
        Map the SPARQL query to the RDF representation of the summary in order to get suggestions.
    \end{block}
}

\defverbatim[colored]\sparql{%
\begin{minted}[linenos,fontsize=\tiny]{sparql}
ASK WHERE {                                                            SELECT ?POF {
  :article1 a :Article .                                                 ?n1 :feature [
                                                                           :label :Article,
                                                                           :type rdf:type
                                                                         ] .

  :article1 :title ?t .                                                  ?e1 :source ?n1 ;
                                                                             :target _:b1 ;
                                                                             :label :title .

  ?i a :Institute .                                                      ?n2 :feature [
                                                                           :label :Institute,
                                                                           :type rdf:type
                                                                         ] .

  ?i :employs ?p .                                                       ?e2 :source ?n2 ;
                                                                             :target ?n3 ;
                                                                             :label ?employs .

  ?p :name "Renaud" .                                                    ?e3 :source ?n3 ;
                                                                             :target _:b2 ;
                                                                             :label :name .

  ?p <                  # POF                                            ?e4 :source ?n3 ;
                                                                             :target _:b3 ;
                                                                             :label ?POF .
}                                                                      }
\end{minted}
}

\begin{frame}[fragile]
    \frametitle{Example: Attribute Suggestions}
    \vspace{1cm}
    \sparql
    \setbeamercovered{invisible}
    \begin{tikzpicture}[remember picture]
        %\draw[step=1cm,gray,ultra thick,overlay] (0,0) grid (12,12);

        \uncover<2->{
            \draw[red,ultra thick,overlay] (.6,1.44) circle (.3);
        }

        \uncover<1-2>{
            \fill[white,ultra thick,overlay] (7,6.95) rectangle (10,6);
        }

        \uncover<1-3>{
            \fill[white,ultra thick,overlay] (7,6) rectangle (10,5);
        }

        \uncover<1-4>{
            \fill[white,ultra thick,overlay] (7,5) rectangle (10,1.8);
        }

        \uncover<1-5>{
            \fill[white,ultra thick,overlay] (7,1.8) rectangle (10,0.8);
            \fill[white,ultra thick,overlay] (7,6.95) rectangle (10,7.4);
            \fill[white,ultra thick,overlay] (7,.8) rectangle (10,0.4);
        }

        \uncover<6->{
            \draw[red,ultra thick,overlay] (9.1,.95) circle (.3);
            \draw[red,ultra thick,overlay] (8.45,7.1) circle (.3);
        }

    \end{tikzpicture}
    \setbeamercovered{transparent}
    %% vim: et:sw=4
\end{frame}

\subsection{Web Data Inspector}

\frame{
    \frametitle{Web Data Inspector}
}
