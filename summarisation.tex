\section{Graph Summarisation}

\subsection{Overview}

\frame{
    \frametitle{Overview}
    \begin{figure}
        \centering
        \begin{subfigure}[b]{.5\textwidth}
            \centering
            \resizebox{\textwidth}{!}{
                \begin{tikzpicture}[->, >=stealth', node distance=2.5cm]
\node[draw, circle] (n0) {$v_0$};
\node[draw, circle, below of = n0] (n01) {Person};
\node[draw, circle, above of = n0] (n02) {Alice};

\node[draw, circle, right of = n0] (n1) {$v_1$};
\node[draw, circle, below of = n1] (n11) {Person};
\node[draw, circle, above of = n1] (n12) {Bob};

\node[draw, circle, right of = n1] (nn) {$v_n$};
\node[draw, circle, below of = nn] (nn1) {Person};
\node[draw, circle, above of = nn] (nn2) {Zed};

\path[every node/.style={fill=white,font=\footnotesize}]
  (n0) edge node {name} (n02)
  (n0) edge node {type} (n01)

  (n1) edge node {name} (n12)
  (n1) edge node {type} (n11)

  (nn) edge node {name} (nn2)
  (nn) edge node {type} (nn1)
  (nn) edge[dotted, thick,-] (n1)
;
\end{tikzpicture}
%% vim: et:sw=4

            }
            \caption{An entity graph about people}
        \end{subfigure}
        \qquad
        \begin{subfigure}[b]{.4\textwidth}
            \centering
            \resizebox{.35\textwidth}{!}{
                \begin{tikzpicture}[->, >=stealth', node distance=2.5cm]
\node[draw, circle] (n0) {$u_0$};
\node[draw, circle, below of = n0] (n01) {Person};
\node[draw, circle, above of = n0] (n02) {$*$};

\path[every node/.style={fill=white,font=\footnotesize}]
  (n0) edge node {name} (n02)
  (n0) edge node {type} (n01)
;
\end{tikzpicture}
%% vim: et:sw=4

            }
            \caption{A possible summary}
        \end{subfigure}
        \caption{Summarising an entity graph}
    \end{figure}
}

\subsection{Model}

\frame[plain]{
    \frametitle{Graph Summary}
    \begin{definition}
        A graph is the \textbf{summary} of another graph if there is a \textbf{homomorphism} from the \emph{latter} to the \emph{former} with respect to a \textbf{relation}.
    \end{definition}
    \hfill\begin{minipage}{\dimexpr\textwidth-1cm}
        \begin{block}{Summarisation Relation}
            \begin{itemize}
                \item Mapping between the nodes of two graphs, based on \emph{features} of the graph
            \end{itemize}
        \end{block}
        \begin{block}{Graph Homomorphism}
            \begin{itemize}
                \item With respect to the summarisation relation, each edge of the graph is mapped to an edge(s) of the summary
            \end{itemize}
        \end{block}
    \end{minipage}
}

\frame{
    \frametitle{Graph Homomorphism}
    \begin{columns}[c]
        \column{.45\textwidth}
        \begin{definition}
            A graph is homomorphic to another if all of its edges do exist in that other graph for a given relation.
        \end{definition}

        \column{.6\textwidth}
        \begin{example}
            \begin{figure}
                \centering
                \resizebox{\textwidth}{!}{
                    \setbeamercovered{invisible}
\begin{tikzpicture}[->,>=stealth',node distance=2.25cm]
    \node [draw,circle] (1) {1};
    \node [draw,circle,below left of = 1] (2) {2};
    \node [draw,circle,below right of = 1] (3) {3};
    \node [draw,circle,above of = 1] (4) {4};

    \node [draw,circle,right of = 1,xshift=2cm,yshift=-1cm] (a) {a};
    \node [draw,circle,above of = a] (b) {b};

    % first edge
    \uncover<2-3>{
        \path[red highlight edge]
            (2) edge (1)
            ;
        \node [draw,circle, red highlight node] (1) {1};
        \node [draw,circle,below left of = 1, red highlight node] (2) {2};
    }
    \uncover<3-3>{
        \path[green highlight edge]
            (1) edge (b)
            (2) edge [bend left=10] (a)
            ;
        \path[red highlight edge]
            (b) edge (a)
            ;
        \node [red highlight node, draw,circle,right of = 1,xshift=2cm,yshift=-1cm] (a) {a};
        \node [red highlight node, draw,circle,above of = a] (b) {b};
    }

    % second edge
    \uncover<4-5>{
        \path[red highlight edge]
            (3) edge (1)
            ;
        \node [draw,circle, red highlight node] (1) {1};
        \node [draw,circle,below right of = 1, red highlight node] (3) {3};
    }
    \uncover<5-5>{
        \path[green highlight edge]
            (1) edge (b)
            (3) edge (a)
            ;
        \path[red highlight edge]
            (b) edge (a)
            ;
        \node [red highlight node, draw,circle,right of = 1,xshift=2cm,yshift=-1cm] (a) {a};
        \node [red highlight node, draw,circle,above of = a] (b) {b};
    }

    % third edge
    \uncover<6-7>{
        \path[red highlight edge]
            (1) edge (4)
            ;
        \node [draw,circle, red highlight node] (1) {1};
        \node [draw,circle,above of = 1, red highlight node] (4) {4};
    }
    \uncover<7-7>{
        \path[green highlight edge]
            (1) edge (b)
            (4) edge (b)
            ;
        \path[red highlight edge]
            (b) edge[loop] (b)
            ;
        \node [red highlight node, draw,circle,above of = a] (b) {b};
    }

    \path[every node/.style={fill=white,font=\footnotesize}]
        (1) edge[->] (4)
        (2) edge (1)
        (3) edge (1)
        (a) edge (b)
        (b) edge[loop] (b)
        (1) edge [dotted] node {$R$} (b)
        (4) edge [dotted] node {$R$} (b)
        (2) edge [dotted,bend left=10] node[near start] {$R$} (a)
        (3) edge [dotted] node {$R$} (a);

\end{tikzpicture}
\setbeamercovered{transparent}
%% vim: et:sw=4

                }
            \end{figure}
        \end{example}
    \end{columns}
}

\subsection{Precise Graph Summary}

\frame{
    \frametitle{Precise Graph Summary}
    \begin{columns}[c]
        \column{.55\textwidth}
        \begin{block}{Inevitability of Precision Loss}
            \begin{itemize}
                \item A summary is precise if every path in the graph do exist
                \item Creating such a summary is impractical due to the data's heterogeneity
            \end{itemize}
        \end{block}

        \column{.5\textwidth}
        \begin{figure}
            \centering
            \resizebox{\textwidth}{!}{
                \begin{tikzpicture}
    \begin{loglogaxis}[
                grid=major,
                        xlabel=number of documents (\emph{log}),
                        ylabel=probability (\emph{log}),
                ]
        \addplot[red, domain=1:100,ultra thick] {0.00257344864/(x^1.913581)};
        \addlegendentry{$\alpha =  1.91$}
        \addplot+[only marks, mark size=1pt,mark=star, blue] table[x index=1,y index=0] {images/summarisation/basic-namespace-stats-probability};
	\end{loglogaxis}
\end{tikzpicture}

            }
            \caption{Ontology probability distribution}
        \end{figure}
    \end{columns}
}

\subsection{Approximate Graph Summaries}

\frame{
    \frametitle{Approximate Graph Summaries}
    \begin{block}{Why approximate a summary ?}
        \begin{itemize}
            \item Web Data is too heterogeneous for its structure to be captured perfectly
            \item Summarisation needs to scale to large graphs
        \end{itemize}
    \end{block}
    \begin{block}{Features}
        \begin{itemize}
            \item Types
            \item Attributes (incoming and/or outgoing)
        \end{itemize} 
    \end{block}

    %\uncover<2->{
        %\begin{block}{Studied Summaries}
            %\begin{columns}[c]
                %\column{.5\textwidth}
                %\begin{itemize}
                    %\item Unique Type Summary
                    %\item Types Summary
                    %\item Attributes Summary
                %\end{itemize} 

                %\column{.6\textwidth}
                %\begin{itemize}
                    %\item Types \& Attributes Summary
                    %\item IO Attributes Summary
                    %\item IO Attributes Types Summary
                %\end{itemize} 
            %\end{columns}
        %\end{block}
    %}
}

\frame[shrink]{
    \frametitle{Example: Types Summary}
    \vspace{2em}
    \begin{description}
        \item[Relation:] set of types associated to a node
    \end{description}
    \vspace{-1em}
    \begin{columns}[c]
        \column{.6\textwidth}
        \begin{figure}[b]
            \centering
            \resizebox{.9\textwidth}{!}{
                \setbeamercovered{invisible}
\begin{tikzpicture}[->,>=stealth',node distance=3.5cm]
    \node [draw,circle] (n2) {$v_1$};
    \node [draw,circle,below of = n2,yshift=-1cm] (n1) {$v_2$};
    \node [draw,circle,right of = n2] (n3) {$v_3$};
    \node [draw,circle,right of = n3] (n6) {$v_6$};
    \node [draw,circle,below of = n6,yshift=-1cm] (n4) {$v_4$};
    \node [draw,circle,below of = n2,yshift=1.25cm] (person) {Person};
    \node [draw,circle,below of = n1] (ren) {Renaud};
    \node [draw,circle,above of = n2] (ste) {St\'ephane};
    \node [draw,circle,right of = n4] (n5) {$v_5$};
    \node [draw,circle,right of = n6] (n7) {$v_7$};
    \node [draw,circle,above right of = n4,xshift=-.7cm] (place) {Place};
    \node [draw,circle,above of = n7,yshift=0cm] (country) {Country};
    \node [draw,circle,below of = n4] (deri) {DERI};
    \node [draw,circle,below of = n5] (fbk) {FBK};
    \node [draw,circle,above left of = n3,xshift=1cm] (city) {City};
    \node [draw,circle,above right of = n3,xshift=-1cm] (gal) {Galway};
    \node [draw,circle,above right of = n7] (rome) {Rome};
    \node [draw,circle,above of = n6] (ire) {Ireland};

    \uncover<2-2>{
        \node [draw,circle,below of = n2,yshift=1.25cm, green highlight node] (person) {Person};
        \path[green highlight edge]
            (person) edge (n2) edge (n1)
            ;
        \node [draw,circle, red highlight node] (n2) {$v_1$};
        \node [draw,circle,below of = n2,yshift=-1cm, red highlight node] (n1) {$v_2$};
    }

    \uncover<3-3>{
        \node [draw,circle,above left of = n3,xshift=1cm, green highlight node] (city) {City};
        \path[green highlight edge]
            (n3) edge (city)
            ;
        \node [draw,circle,right of = n2, red highlight node] (n3) {$v_3$};
    }

    \uncover<4-4>{
        \node [draw,circle,above right of = n4,xshift=-.7cm, green highlight node] (place) {Place};
        \path[green highlight edge]
            (n4) edge (place)
            (n5) edge (place)
            ;
        \node [draw,circle,below of = n6,yshift=-1cm, red highlight node] (n4) {$v_4$};
        \node [draw,circle,right of = n4, red highlight node] (n5) {$v_5$};
    }

    \uncover<5-5>{
        \node [draw,circle,above of = n7,yshift=0cm, green highlight node] (country) {Country};
        \node [draw,circle,above right of = n4,xshift=-.7cm, green highlight node] (place) {Place};
        \path[green highlight edge]
            (place) edge (n6) edge (n7)
            (country) edge (n6) edge (n7)
            ;
        \node [draw,circle,right of = n3, red highlight node] (n6) {$v_6$};
        \node [draw,circle,right of = n6, red highlight node] (n7) {$v_7$};
    }

    \uncover<6-6>{
        \node [draw,circle,below of = n1, red highlight node] (ren) {Renaud};
        \node [draw,circle,above of = n2, red highlight node] (ste) {St\'ephane};
        \node [draw,circle,below of = n4, red highlight node] (deri) {DERI};
        \node [draw,circle,below of = n5, red highlight node] (fbk) {FBK};
        \node [draw,circle,above right of = n3,xshift=-1cm, red highlight node] (gal) {Galway};
        \node [draw,circle,above right of = n7, red highlight node] (rome) {Rome};
        \node [draw,circle,above of = n6, red highlight node] (ire) {Ireland};
    }

    \uncover<7-7>{
        \node [draw,circle, red highlight node] (n2) {$v_1$};
        \node [draw,circle,above of = n2, red highlight node] (ste) {St\'ephane};
        \draw [red highlight edge] (n2) edge node[fill=white] {name} (ste);
    }

    \uncover<8-8>{
        \node [draw,circle, red highlight node] (n2) {$v_1$};
        \node [draw,circle, red highlight node] (n2) {$v_1$};
        \draw [red highlight edge] (n2) edge node[fill=white] {lives} (n3);
    }

    \path
        (n7) edge node[fill=white] {capital} (rome)
        (n6) edge node[fill=white] {label} (ire)

        (n3) edge node[fill=white] {label} (gal)
        (n3) edge node[fill=white] {type} (city)

        (n4) edge node[fill=white] {label} (deri)
        (n5) edge node[fill=white] {label} (fbk)

        (n6) edge node[fill=white] {type} (country)
        (n7) edge node[fill=white] {type} (country)

        (n4) edge node[fill=white] {type} (place)
        (n5) edge node[fill=white] {type} (place)
        (n6) edge node[fill=white] {type} (place)
        (n7) edge node[fill=white] {type} (place)

        (n3) edge node[fill=white] {location} (n6)
        (n4) edge node[fill=white] {location} (n6)
        (n5) edge node[fill=white] {location} (n7)
        (n1) edge node[fill=white] {name} (ren)
        (n2) edge node[fill=white] {name} (ste)
        (n1) edge node[fill=white] {type} (person)
        (n2) edge node[fill=white] {type} (person)
        (n1) edge node[fill=white] {lives} (n3)
        (n1) edge node[fill=white] {works} (n4)
        (n2) edge node[fill=white] {lives} (n3)
        (n2) edge[near end] node[fill=white] {works} (n4)
    ;
\end{tikzpicture}
\setbeamercovered{transparent}
%% vim: et:sw=4

            }
            \caption{An entity graph describing people, places, and their relationships.}
        \end{figure}

        \column{.4\textwidth}
        \begin{figure}[b]
            \centering
            \resizebox{.9\textwidth}{!}{
                \setbeamercovered{invisible}
\begin{tikzpicture}[->,>=stealth',node distance=4cm]
    \uncover<2-2,7-8>{
        \node [draw,circle, red highlight node] (h1) {$S_1$};
    }
    \uncover<2->{
        \node [draw,circle] (h1) {$S_1$};
    }

    \uncover<3-3,8-8>{
        \node [draw,circle,above right of = h1, red highlight node] (h2) {$S_2$};
    }
    \uncover<3->{
        \node [draw,circle,above right of = h1] (h2) {$S_2$};
    }

    \uncover<4-4>{
        \node [draw,circle,below right of = h1, red highlight node] (h3) {$S_3$};
    }
    \uncover<4->{
        \node [draw,circle,below right of = h1] (h3) {$S_3$};
    }

    \uncover<5-5>{
        \node [draw,circle,above right of = h3, red highlight node] (h4) {$S_4$};
    }
    \uncover<5->{
        \node [draw,circle,above right of = h3] (h4) {$S_4$};
    }

    \uncover<2-2>{
        \node [draw,circle,above of = h1, green highlight node] (person) {Person};
    }
    \uncover<2->{
        \node [draw,circle,above of = h1] (person) {Person};
    }

    \uncover<3-3>{
        \node [draw,circle,above of = h2, green highlight node] (city) {City};
    }
    \uncover<3->{
        \node [draw,circle,above of = h2] (city) {City};
    }

    \uncover<5-5>{
        \node [draw,circle,above of = h4, green highlight node] (country) {Country};
    }
    \uncover<5->{
        \node [draw,circle,above of = h4] (country) {Country};
    }

    \uncover<4-5>{
        \node [draw,circle,below right of = h4,xshift=-1cm, green highlight node] (place) {Place};
    }
    \uncover<4->{
        \node [draw,circle,below right of = h4,xshift=-1cm] (place) {Place};
    }

    \uncover<6-7>{
        \node [draw,circle,below of = h2,yshift=1cm, red highlight node] (c2) {$\varnothing$};
    }
    \uncover<6->{
        \node [draw,circle,below of = h2,yshift=1cm] (c2) {$\varnothing$};
    }

    %%%
    %%% Edges
    %%%
    \uncover<2-2>{
        \draw [green highlight edge] (h1) edge node[fill=white] {type} (person);
    }
    \uncover<2->{
        \draw (h1) edge node[fill=white] {type} (person);
    }

    \uncover<3-3>{
        \draw [green highlight edge] (h2) edge node[fill=white] {type} (city);

    }
    \uncover<3->{
        \draw (h1) (h2) edge node[fill=white] {type} (city);
    }

    \uncover<4-4>{
        \draw [green highlight edge] (h3) edge node[fill=white] {type} (place);
    }
    \uncover<4->{
        \draw (h3) edge node[fill=white] {type} (place);
    }

    \uncover<5-5>{
        \draw [green highlight edge] (h4) edge node[fill=white] {type} (place);
    }
    \uncover<5->{
        \draw (h4) edge node[fill=white] {type} (place);
    }

    \uncover<5-5>{
        \draw [green highlight edge] (h4) edge node[near end,fill=white] {type} (country);
    }
    \uncover<5->{
        \draw (h4) edge node[near end,fill=white] {type} (country);
    }

    \uncover<7-7>{
        \draw [red highlight edge] (h1) edge node[fill=white] {name} (c2);
    }
    \uncover<7->{
        \draw (h1) edge node[fill=white] {name} (c2);
    }

    \uncover<8-8>{
        \draw [red highlight edge] (h1) edge[bend left] node[fill=white] {lives} (h2);
    }
    \uncover<8->{
        \draw (h1) edge[bend left] node[fill=white] {lives} (h2);
    }

    \uncover<9->{
        \path
            (h3) edge node[fill=white] {label} (c2)
            (h2) edge node[fill=white] {label} (c2)
            (h4) edge[bend right] node[fill=white] {label} (c2)
            (h4) edge[bend left] node[fill=white] {capital} (c2)

            (h3) edge node[fill=white] {type} (place)
            (h1) edge[bend right] node[fill=white] {works} (h3)
            (h2) edge[bend left] node[fill=white] {location} (h4)
            (h3) edge[bend right] node[fill=white] {location} (h4)
            ;
    }
\end{tikzpicture}
\setbeamercovered{transparent}
%% vim: et:sw=4

            }
            \caption{Types summary of the entity graph.}
        \end{figure}
    \end{columns}
}

\subsection{Implementations}

\frame[shrink]{
    \frametitle{Implementations}
    \vspace{.5cm}
    \begin{block}{Algorithm}
        \begin{enumerate}
            \item Entity description (group operator)
            \item Nodes mapping (object invention)
            \item Edges materialization (join operator)
        \end{enumerate}
    \end{block}
    \uncover<2->{
        \begin{columns}[c]
            \column{.5\textwidth}
            \begin{block}{SPARQL}
                \begin{alert}{Performance:}
                    Timeout for graphs above 20M triples.
                \end{alert} 
                \\
                \begin{alert}{Pros:}
                    Leverage endpoint for optimizing queries.
                \end{alert}
                \\
                \begin{alert}{Const:}
                    Bounded by the expressivity of SPARQL.
                \end{alert}
            \end{block}

            \column{.5\textwidth}
            \begin{block}{MapReduce}
                \begin{alert}{Performance:}
                    Scale to large graphs up to 50B triples.
                \end{alert} 
                \\
                \begin{alert}{Pros:}
                    More possiblities for optimisations.
                \end{alert}
                \\
                \begin{alert}{Cons:}
                    Cluster maintenance.
                \end{alert}
            \end{block}
        \end{columns}
    }
}
%% vim: et:sw=4
